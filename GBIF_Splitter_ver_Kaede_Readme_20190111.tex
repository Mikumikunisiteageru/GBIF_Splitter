\documentclass[a4paper, utf-8]{ctexart}
\usepackage[margin=25mm, top=10mm, bottom=10mm]{geometry}
\usepackage{mdwlist}
\CTEXsetup[format={\Large\bfseries}]{section}

\catcode`_=13
\def_{\_}

\title{\texttt{GBIF_Splitter_ver_Kaede} 使用说明}
	\author{杨宇昌\footnote{Email: \texttt{yang.yc.allium@gmail.com}}}
	\date{2019年1月11日}

\begin{document}
\maketitle

\section{用途}

将 GBIF 网站的 Occurence Data 按国家或地区划分成较小的文件,每条记录保留以下字段:
\begin{itemize*}
	\item \verb|gbifID|:GBIF 记录编号
	\item \verb|family|:科名
	\item \verb|genus|:属名
	\item \verb|species|:种名
	\item \verb|infraspecificEpithet|:种下阶元的加词
	\item \verb|taxonRank|:分类阶元等级
	\item \verb|scientificName|:学名
	\item \verb|countryCode|:国家代码
	\item \verb|locality|:地点
	\item \verb|decimalLatitude|:纬度,北纬为正,南纬为负,用十进制表示,单位为角度
	\item \verb|decimalLongitude|:经度,东经为正,西经为负,用十进制表示,单位为角度
	\item \verb|elevation|:海拔高度,单位为米
\end{itemize*}
这些较小的文件以格式 \verb|GBIF_ver_Kaede_|\textit{countryCode}\verb|.txt| 命名,其中 \textit{countryCode} 表示记录中 \verb|countryCode| 字段的非空值,如中国大陆的对应值为 \verb|CN|;\verb|countryCode| 字段空白的记录如存在,则放在文件 \verb|GBIF_ver_Kaede_Unlabeled.txt| 中。程序同时还会生成名为 \verb|GBIF_ver_Kaede_Report.txt| 的文件,当中保存了前述每个小文件中的记录数。

\section{环境要求}

Windows 10 (64x) 系统下可直接运行。要在其他系统中运行,需自行编译\footnote{源代码见 \texttt{https://github.com/Mikumikunisiteageru/GBIF_Splitter/}}。


\section{使用方法}

\begin{enumerate*}
	\item 在 GBIF 网站上选取需要的 Occurence Data,以 CSV 格式下载;
	\item 打开下载的 ZIP 文件,将包含的 CSV 文件解压到任意目录下,重命名为 \verb|GBIF_Global.txt|;
	\item 将 \verb|GBIF_Splitter_ver_Kaede.exe| 移动到同一目录中后运行。
\end{enumerate*}

\section{文件列表}

\begin{itemize*}
	\item \verb|GBIF_Splitter_ver_Kaede.exe|:程序文件
	\item \verb|GBIF_Splitter_ver_Kaede_readme.pdf|:使用说明(即本文件)
	\item \verb|GBIF_Global.txt|:样例输入,此处为 GBIF 网站上 \textit{Euptelea} 属的 Occurence Data
\end{itemize*}

\end{document}

